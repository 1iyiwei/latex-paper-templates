
%% the following commands are plagiarized from Ravi
\newcommand{\pcline }{\rule{0in}{0.0in}    }  % Spacing for pseudo-code.
\newcommand{\pctab  }{\hspace{0.10in}      }  % Pseudo-code indentation.
\newcommand{\pcbigtab  }{\hspace{1.75in}    }  % Pseudo-code indentation.
\newcommand{\pcasgn }{\mbox{$\leftarrow$}  }  % Pseudo-code assignment operator
%%\newcommand{\pcgreater}{\mbox{$\leftarrow$} } % Pseudo-code bigger operator
\newcommand{\pcomment}[1]{\mbox{// {\it #1}}}  % Pseudo-code comments.
\newcommand{\pckeyword}[1]{\mbox{{\bf #1}}}  % Pseudo-code keywords.
\newcommand{\pcspace}{\hspace{0.10in}}

\newcommand{\pcode}[1]{
    \footnotesize
    \vspace{-0.1in}
    \begin{minipage}{100in} % The width argument will be ignored:
                            % See lamport p. 99
    \begin{tabbing} \hspace{0.10in} \= \pctab \= \pctab \= \pctab \= \pcbigtab \= \\
        #1
    \end{tabbing}
    \end{minipage}
%    \vspace{0.15in}
    }
%% enf of the pseudo code environment

\newcommand{\tabtextsize}{\small}
\newcommand{\figtext}[1]{{\footnotesize #1}}
%%\newcommand{\Caption}[1]{\caption{\small {\em #1}}}
%%\newcommand{\Caption}[1]{\caption{{\em #1}}}
\newcommand{\Caption}[2]{\caption[#1]{{\footnotesize #1} {\footnotesize #2}}}
%%\newcommand{\Caption}[1]{\caption{\it \small {#1}}}

\newcommand{\invisible}[1]{{\color{white} #1}}
\newcommand{\emacsquote}[1]{{``#1''}}

\definecolor{SithColor}{rgb}{0.7,0,0} % color for Sith
\newcommand{\liyi}[1]{{\color{SithColor} Li-Yi: #1 $\qed$}}
\definecolor{ConsularColor}{rgb}{0,0.4,0} % color for the Jedi Consulars (e.g. Yoda)
\definecolor{GuardianColor}{rgb}{0,0,0.8} % color for the Jedi Guardians (e.g. Obiwan)
\newcommand{\obiwan}[1]{{\color{GuardianColor} Obiwan: #1 $\qed$}}
\newcommand{\yoda}[1]{{\color{ConsularColor} Yoda: #1 $\qed$}}
\newcommand{\warning}[1]{{\it\color{red} #1}}
\newcommand{\note}[1]{{\it\color{blue} #1}}
\newcommand{\nothing}[1]{}
\newcommand{\passthrough}[1]{#1}
\definecolor{AudioColor}{rgb}{0.56,0.34,0.62}
\newcommand{\audio}[1]{{\color{AudioColor} Audio: #1}}

\newcommand{\psdraftboxDefault}{\psnodraftbox}

\definecolor{figred}{rgb}{1,0,0}
\definecolor{figgreen}{rgb}{0,0.6,0}
\definecolor{figblue}{rgb}{0,0,1}
\definecolor{figpink}{rgb}{1,0.63,0.63}

%\renewcommand{\warning}[1]{}

\ifthenelse{\equal{\final}{1}}
{
\renewcommand{\liyi}[1]{}
\renewcommand{\obiwan}[1]{}
\renewcommand{\yoda}[1]{}
\renewcommand{\warning}[1]{}
\renewcommand{\note}[1]{}
}
{}

\newcommand{\pseudocode}{Pseudocode}
\floatstyle{plain}
\newfloat{algorithm}{tbhp}{lop}
\floatname{algorithm}{\pseudocode}

\newcommand{\filename}[1]{\url{#1}}
\newcommand{\foldername}[1]{\url{#1}}
\newcommand{\email}[1]{\url{#1}}
